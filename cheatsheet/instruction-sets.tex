\section{Instruction Sets}

\emph{Arithmetic Operations}: treat operands as numbers; consider signs;
(\eg arithmetic shift = multiplication/division by 2, sign bit preserved).\\
\emph{Logical Operations}: treat operands as bit patterns; discard bits shifted out;
replenish new bits with 0.\\
\emph{Rotate Operations}: put bits shifted out back into the other end of the number;
are logical operations.\\

\subsection*{Instruction Operands}
Op\# \hfill Symbolic \hfill Interpret\\
3 \hspace*{0.2\linewidth} \texttt{OP A, B, C} \hfill \texttt{A$\gets$B OP C}\\
2 \hspace*{0.2\linewidth} \texttt{OP A, B} \hfill \texttt{A$\gets$A OP B}\\
1 \hspace*{0.2\linewidth} \texttt{OP A} \hfill \texttt{AC$\gets$AC OP A}\\
0 \hspace*{0.2\linewidth} \texttt{OP} \hfill \texttt{T$\gets$(T-1) OP T}\\

\subsection*{Registers}
\emph{General Purpose Registers}: can be used for whatever reason\\
\emph{Dedicated Purpose Registers}: have a specific purpose (\eg PC, IR, SP, processor status word - PSW, flag)

\subsection*{Data Types}

\subsubsection*{Basic Data Types}
Typical lengths: 8, 16, 32, 64 bits\\
\emph{Numeric}: integer, floating point;\\
\emph{Non-numeric}: character, binary data;

\subsubsection*{MIPS Architecture}
(family of RISC, not ARM nor x86)\\
9 basic types: \begin{enuminline}
    \item (un)/signed bytes;
    \item (un)/signed half-words;
    \item (un)/signed words;
    \item double-words;
    \item single-precision floating point (32 bits);
    \item double-precision floating point (64 bits);
\end{enuminline}

\subsubsection*{ARM Architecture}
Supported lengths: \begin{enuminline}
    \item byte (8 bits);
    \item half-word (16 bits);
    \item word (32 bits);
\end{enuminline}\\
Only unsigned integers, nonnegative integers, and 2's comp integers.\\
No floating point by hardware, must be emulated.