\subsection{Cache Memory}

Cache memory is transparent (hidden) to the software and is managed by the hardware.
It stores copies of frequently accessed data to speed up subsequent access to that data.

There can be one or more layers between the CPU and main memory.
The transfer between the CPU and L1 cache is the fastest, and the speed decreases
as the distance from the CPU increases.

\subsubsection{Cache Memory Organisation}

A word-addressable main memory with $n$-bit addresses has $2^n$ words. Divide the
main memory into blocks of $K$ words each, then the memory has $M=\frac{2^n}{K}$ blocks.
Suppose the cache has $m$ blocks, called \textbf{lines}. Generally, we have $m \ll M$.
Each line has $K$ words, a tag, and several control bits. The length of the line,
excluding the tag and the control bits is called the \textbf{line size} or
\textbf{block length}.

\subsubsection{Address Mapping}