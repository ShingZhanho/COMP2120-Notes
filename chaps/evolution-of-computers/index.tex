\section{Evolution of Computers}

\subsection{Bench-marking Performance}

\begin{definition}[Clock Speed]
    The pulse frequency ($f$) by the clock, measured in cycles per
    second, or Hertz (Hz). Also known as clock rate, clock speed.
    A cycle is technically a synchronised pulse.
\end{definition}

Programs consist of instructions, and each instruction has several cycles.
For example, a classic RISC\footnote{RISC: Reduced Instruction Set Computing} pipeline
may have these 5 stages:
\begin{itemize}
    \item Fetch (IF)
    \item Decode (ID)
    \item Execution/Effective Address (EX)
    \item Memory Access (MEM)
    \item Writeback (WB)
\end{itemize}

\begin{definition}[Average Cycles Per Instruction (CPI)] \label{def:average-cpi}
\begin{equation*}
    \text{Average CPI} = \frac{\sum_i \text{CPI}_i\times I_i}{I_c}
\end{equation*}
where $I_i$ is the number of instructions of type $i$, and $I_c = \sum_i I_i$.
\end{definition}

\begin{definition}[Processor Time ($T$)]\label{def:processor-time}
\begin{equation*}
    T = \frac{I_c \times \text{CPI}}{f}
\end{equation*}
\end{definition}

\begin{definition}[Million Instructions Per Second (MIPS)]\label{def:mips}
\begin{equation*}
    \text{MIPS} = \frac{f}{\text{CPI}\times 10^6}
\end{equation*}
\end{definition}

Other useful metrics: Million Floating Point Operations Per Second (MFLOPS).

\begin{remark}
    MIPS and MFLOPS may not accurately reflect the performance of a computer.
    A better approach is to measure the time required to do some real jobs.
    Standard Performance Evaluation Corporation (SPEC) benchmarks are used for this.
\end{remark}