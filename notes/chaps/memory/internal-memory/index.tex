\subsection{Internal Memory}

This section presents some common CMOS\footnote{Complementary Metal-Oxide-Semiconductor} memory
technologies and their characteristics.

\subsubsection{Read-Only Memory (ROM)}

\begin{itemize}
    \item Fabricated like integrated circuit chips.
    \item Non-volatile memory, i.e., retains data even when power is turned off.
    \item \textbf{Programmable ROM (PROM)}: Can be programmed once. Usually done by supplier
        or customer after chip is manufactured.
    \item Content cannot be changed.
\end{itemize}

\subsubsection{Read-Mostly Memory}

\begin{itemize}
    \item \textbf{Erasable Programmable ROM (EPROM)}: Can be erased and reprogrammed multiple times.
        Erasing is done by exposing the chip to ultraviolet light for a specified time.
    \item \textbf{Electrically Erasable PROM (EEPROM)}: Can be erased and reprogrammed in place.
        Only the bytes addressed are changed. Erasing process is slow.
\end{itemize}

\subsubsection{Flash Memory}

\begin{itemize}
    \item Non-volatile memory.
    \item Between EPROM and EEPROM.
    \item Faster write than EEPROM.
    \item Have limited number of write cycles.
    \item Usage: USB drives, SSD, storage of BIOS (in recent years).
\end{itemize}

\subsubsection{Random Access Memory (RAM)}

\begin{table}[H]
    \centering
    \begin{tabular}{|l|l|l|}
    \hline
    \textbf{Characteristic} & \textbf{Dynamic RAM}                          & \textbf{Static RAM}        \\ \hline
    Storage Technology      & Use transistors to store electric charges.    & Use logic gates (latches). \\ \hline
    Refreshing              & Required (every few ms due to leaking charge) & Not required               \\ \hline
    Speed                   & Slower (delay due to capacitance)             & Faster                     \\ \hline
    Usage                   & Main memory                                   & Cache memory               \\ \hline
    Cost                    & Cheap                                         & Very expensive             \\ \hline
    \end{tabular}
\end{table}

\subsubsection{Comparison of Memory Types}

\begin{table}[H]
    \centering
    \begin{tabular}{|l|l|l|l|l|}
    \hline
    \textbf{Memory Type} & \multicolumn{1}{c|}{\textbf{Category}} & \multicolumn{1}{c|}{\textbf{Erasure}}                               & \multicolumn{1}{c|}{\textbf{\begin{tabular}[c]{@{}c@{}}Write\\ Mechanism\end{tabular}}} & \multicolumn{1}{c|}{\textbf{Volatility}} \\ \hline
    RAM                  & Read-write                             & \begin{tabular}[c]{@{}l@{}}Electrically,\\ byte-level\end{tabular}  & Electrically                                                                            & Volatile                                 \\ \hline
    ROM                  & \multirow{2}{*}{Read-only}             & \multirow{2}{*}{Not possible}                                       & Masks                                                                                   & \multirow{5}{*}{Nonvolatile}             \\ \cline{1-1} \cline{4-4}
    PROM                 &                                        &                                                                     & \multirow{4}{*}{Electrically}                                                           &                                          \\ \cline{1-3}
    EPROM                & \multirow{3}{*}{Read-mostly}           & \begin{tabular}[c]{@{}l@{}}UV light,\\ chip-level\end{tabular}      &                                                                                         &                                          \\ \cline{1-1} \cline{3-3}
    EEPROM               &                                        & \begin{tabular}[c]{@{}l@{}}Electrically,\\ byte-level\end{tabular}  &                                                                                         &                                          \\ \cline{1-1} \cline{3-3}
    Flash                &                                        & \begin{tabular}[c]{@{}l@{}}Electrically,\\ block-level\end{tabular} &                                                                                         &                                          \\ \hline
    \end{tabular}
\end{table}

\subsubsection{Bench-marking Memory Performance}

Key metrics for memory performance include:
\begin{itemize}
    \item \textbf{Access Time}: Time taken to read/write data.
    \item \textbf{Bandwidth/Transfer Rate}: Rate at which data can be read/written.
    \item \textbf{Memory Cycle Time}: ($\text{Access Time} + \text{Transfer Time}$).
\end{itemize}

\begin{example}
    A two-level memory system has an upper level with $0.01\mu \text{s}$ access time and
    a lower level with $0.1\mu \text{s}$. The upper level has a hit rate of $95\%$. The
    average time to access memory is:
    \begin{equation*}
        0.01 \times 0.95 + (0.01 + 0.1) \times 0.05 = 0.015\mu \text{s}
    \end{equation*}
\end{example}

\subsubsection{Error Detection and Correction}

Errors may arise from various sources such as spike in voltage (lightning), electromagnetic
interference (cosmic ray), power supply problems, etc. Extra bits are required to detect errors.
(Usually in secondary storage devices but not in main memory as it is volatile and less likely
to have errors.)

Common error detection and correction methods include:
\begin{itemize}
    \item \textbf{Parity Bit}: Extra bit added to data to make number of 1s even/odd.
        (Cannot be used to correct errors. Cannot detect 2$n$-bit errors.)
    \item \textbf{Hamming Code}
    \item \textbf{Repetition Code}
\end{itemize}