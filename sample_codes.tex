\section{Sample Chapter}
Let's dive right in!

\subsection{Some Definitions}
\begin{definition}
The \textbf{derivative} of a function $f:I\to\R$ at $a\in I$ is given by:
\begin{equation*}
    f'(x)=\lim\limitsx\frac{f(x)-f(a)}{x-a}
\end{equation*}
\end{definition}

\begin{center}
You know those awesome commutative diagrams?

\begin{tikzcd}
A \arrow[r,"p"] \arrow[d,red,"q"'] & B \arrow[d,"r" red] \\
C \arrow[r,red,"s"' blue] & D
\end{tikzcd}

The derivative has \emph{nothing} to do with them!
\end{center}

\begin{proposition}\label{diffcont}
If $f$ is differentiable at $a$, then $f$ is continuous at $a$.
\end{proposition}
\begin{proof}
Exercise (but only because this is a template).
\end{proof}

The converse of \Cref{diffcont} is not true in general.

\begin{examples}\leavevmode % This is needed to start the list in the next line so it won't be misaligned
\begin{enumerate}
    \item $f(x)=\abs{x}$
    \item $f(x)=\begin{cases} \sin(x) & x\ge 0 \\ 0 & x<0 \end{cases}$
\end{enumerate}
\end{examples}

\begin{theorem}
The following statements are true:
\begin{enumerate}
    \item First statement
    \item Second statement
\end{enumerate}
\end{theorem}
\begin{proof}% For some reason, the proof environment does not need \leavevmode
\begin{enumerate}
    \item Trivial.
    \item Trivial.\qedhere % qedhere is to place the qed symbol here instead of in the next line
\end{enumerate}
\end{proof}

\begin{corollary}
We are both very lucky to have each other as a collaborator.
\end{corollary}
\begin{proof}
We simply note that:
\begin{equation*}
    \frac{1}{1}+\frac{1}{1}\gg\frac{1}{1} \qedhere
\end{equation*}
\end{proof}
\begin{remark}
This corollary is also obvious from empirical evidence.
\end{remark}

\begin{lemma}
$(a+b)^2=a^2+2ab+b^2$
\end{lemma}
\begin{proof}
Expand the left side.
\end{proof}
\begin{remarks}\leavevmode
\begin{enumerate}
    \item It's also kind of obvious.
    \item No extra points for guessing what $(a-b)^2$ is.
\end{enumerate}
\end{remarks}

\begin{example}
$(2+4)^2=2^2+2\cdot 2\cdot 4+4^2=36$
\end{example}

\begin{theorem}[Pythagoras' Theorem]\label{pythagoras}
If $c$ is the hypotenuse of a right triangle and $a$ and $b$ are the other two sides, then $a^2+b^2=c^2$.
\end{theorem}
\begin{proof}
Draw a picture and convince yourself.
\end{proof}

\hyperref[pythagoras]{Pythagoras' theorem} helps motivate the study of metric spaces, which you can learn about in \cite{sekhon}.\\

A lot of nice integrals can be computed using the residue theorem, see \cite[Section 5.2]{taylor}.


\begin{verbatim*}
    \section{Sample Chapter}
Let's dive right in!

\subsection{Some Definitions}
\begin{definition}
The \textbf{derivative} of a function $f:I\to\R$ at $a\in I$ is given by:
\begin{equation*}
    f'(x)=\lim\limitsx\frac{f(x)-f(a)}{x-a}
\end{equation*}
\end{definition}

\begin{center}
You know those awesome commutative diagrams?

\begin{tikzcd}
A \arrow[r,"p"] \arrow[d,red,"q"'] & B \arrow[d,"r" red] \\
C \arrow[r,red,"s"' blue] & D
\end{tikzcd}

The derivative has \emph{nothing} to do with them!
\end{center}

\begin{proposition}\label{diffcont}
If $f$ is differentiable at $a$, then $f$ is continuous at $a$.
\end{proposition}
\begin{proof}
Exercise (but only because this is a template).
\end{proof}

The converse of \Cref{diffcont} is not true in general.

\begin{examples}\leavevmode % This is needed to start the list in the next line so it won't be misaligned
\begin{enumerate}
    \item $f(x)=\abs{x}$
    \item $f(x)=\begin{cases} \sin(x) & x\ge 0 \\ 0 & x<0 \end{cases}$
\end{enumerate}
\end{examples}

\begin{theorem}
The following statements are true:
\begin{enumerate}
    \item First statement
    \item Second statement
\end{enumerate}
\end{theorem}
\begin{proof}% For some reason, the proof environment does not need \leavevmode
\begin{enumerate}
    \item Trivial.
    \item Trivial.\qedhere % qedhere is to place the qed symbol here instead of in the next line
\end{enumerate}
\end{proof}

\begin{corollary}
We are both very lucky to have each other as a collaborator.
\end{corollary}
\begin{proof}
We simply note that:
\begin{equation*}
    \frac{1}{1}+\frac{1}{1}\gg\frac{1}{1} \qedhere
\end{equation*}
\end{proof}
\begin{remark}
This corollary is also obvious from empirical evidence.
\end{remark}

\begin{lemma}
$(a+b)^2=a^2+2ab+b^2$
\end{lemma}
\begin{proof}
Expand the left side.
\end{proof}
\begin{remarks}\leavevmode
\begin{enumerate}
    \item It's also kind of obvious.
    \item No extra points for guessing what $(a-b)^2$ is.
\end{enumerate}
\end{remarks}

\begin{example}
$(2+4)^2=2^2+2\cdot 2\cdot 4+4^2=36$
\end{example}

\begin{theorem}[Pythagoras' Theorem]\label{pythagoras}
If $c$ is the hypotenuse of a right triangle and $a$ and $b$ are the other two sides, then $a^2+b^2=c^2$.
\end{theorem}
\begin{proof}
Draw a picture and convince yourself.
\end{proof}

\hyperref[pythagoras]{Pythagoras' theorem} helps motivate the study of metric spaces, which you can learn about in \cite{sekhon}.\\

A lot of nice integrals can be computed using the residue theorem, see \cite[Section 5.2]{taylor}.
\end{verbatim*}